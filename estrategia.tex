%!TEX root=main.tex
La valoración de una patente de invención presenta varios desafíos como se plantea en \url{http://users.ox.ac.uk/~mast0140/EJWP0599.pdf}. El primer de estos es que la comparación de su valor con el de otras patentes similares es muy riesgosa y suele no entregar resultados confiables. El segundo es que, naturalmente al ser un producto nuevo, existe muy poca información sobre cómo adoptarán el producto los consumidores y también el impacto que tendrá una determinada estrategia de marketing. Luego, como ni el método por comparación de mercado ni por múltiplos son válidos se debe valorizar mediante la estimación de los flujos actualizados para calcular el Valor Actualizado Neto (VAN) que le generará al dueño de la patente el poseer esta invención. Como al Sr. Stuardo sólo le interesa dedicarse a la producción de alcuzas basadas en el SAI, se decidió formular un negocio basado en una empresa que represente a un posible comprador que incorporará este diseño en su surtido de productos.
