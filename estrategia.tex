%!TEX root=main.tex
La valoración de una patente de invención presenta varios desafíos como se plantea en \cite{pitkethly1997valuation}. El primer de estos es que la comparación de su valor con el de otras patentes similares es muy riesgosa y suele no entregar resultados confiables. El segundo es que, naturalmente al ser un producto nuevo, existe muy poca información sobre cómo adoptarán el producto los consumidores y también el impacto que tendrá una determinada estrategia de marketing. Luego, como ni el método por comparación de mercado ni por múltiplos son válidos se debe valorizar mediante la estimación de los flujos actualizados para calcular el Valor Actualizado Neto (VAN) que le generará al dueño de la patente el poseer esta invención. Como al Sr. Stuardo sólo le interesa dedicarse a la producción de alcuzas basadas en el SAI, se decidió formular un negocio basado en una empresa que represente a un posible comprador que incorporará este diseño en su surtido de productos.


Las características de este negocio son las siguientes:
\begin{itemize}
\item La empresa cuenta con una base de operaciones ya formada, esto significa que no existirán costos fijos relacionados con la administración aparte del esfuerzo extra de la inclusión del producto.
\item La empresa posee la capacidad de incluir a la alcuza dentro de su línea de producción.
\item La empresa cuenta con el personal adecuado para ingresar este producto al mercado.
\item La empresa posee el capital para invertir en el negocio sin la necesidad de endeudarse.
\item La empresa será operativa durante todo el ciclo de vida del producto.
\item La empresa posee bodegas para inventariar la alcuza, por lo que los costos corresponden a un porcentaje de esfuerzo adicional y no a la compra de una nueva bodega.
\item  La empresa cuenta con una logística de despacho ya operativa y esta puede incluir nuevos ítems agregando un costo adicional porcentual, por lo que no es necesario la compra de nuevas flotas ni personal.
\end{itemize}

El mercado en el que se engloba esta empresa es el mercado de las alcuzas. El producto final de este diseño entrará en la categoría de los artículos de cocina y mesa: alcuzas y especieros. Los mercados a los cuales sirve esta industria corresponden a distintos segmentos, siendo principalmente de nicho, estos incluyen:
\begin{itemize}
\item Doméstico: Personas que compran artículos (alcuzas) para su casa.
\item Industria gastronómica: Corresponde a cocinerías, restaurantes, casinos y hoteles,
\item Industria de proveedores para la gastronomía: Muchos proveedores de insumos (aceites, vinagres, especies) poseen sus propias alcuzas que venden en conjunto al insumo como un valor agregado a través de distintos canales de venta.
\item Minoristas: Empresas que venden productos al detalle, tanto a un público general como a uno específico (artículos especializados, de diseñador, etc).
\end{itemize}

La mayor parte de la oferta de alcuzas existente en las tiendas comerciales viene de las importaciones, siendo China el exportador principal de este tipo de producto, lo que se confirmó realizando una investigación del origen de los productos en el mercado. Un ejemplo de esto son las empresas de retail, las que compran una gama de diseños variada y los rotan cada cierto número de semanas, el que depende de cada casa comercial.

El modelo de compra de la industria gastronómica se caracteriza principalmente por la compra anual de estos productos y guardar en inventarios los repuestos que podrían necesitar durante el año. La compra de estos son a las casas comerciales que venden estos productos al por mayor.

Dentro del mercado de alcuzas se distinguen dos mercados: el mercado de envases y el mercado de aceites.

\begin{itemize}
\item \textbf{Mercado de Envases}
Uno de los potenciales clientes compradores de este diseño son las empresas que elaboran envases de vidrio ya que poseen las competencias necesarias para el desarrollo de esta alcuza. Este segmento posee una alta variedad de productos, entre ellos se puede mencionar envases de vino, cervezas, licores, alimento, entre otros. Dicho tipo de organización podría estar interesada en incorporar este producto entre sus líneas de producción. Las principales empresas chilenas de esta categoría son Cristoro y Cristal Chile, y la multinacional Veralia, pero existen muchas empresas de menor tamaño como Cristalart, Envasesdevidrio, etc.

\item \textbf{Mercado de Aceites}
Uno de los mercados en el cuál este producto podría explotarse de mejor manera son los relacionados a producción de aceite, ya que la mayoría de estas, en su sección gourmet, utilizan envases de vidrio para embotellar sus productos. Un ejemplo de este es el mercado de aceite de oliva que provee un nicho importante debido al crecimiento de la producción de aceite de oliva para su exportación, en aquel contexto un envase de alcuza integrada podría otorgarle características diferenciadoras a una empresa de aceite de oliva al momento de presentar su producto a nivel internacional.
\end{itemize}

\subsubsection{Descripción de las decisiones estratégicas a evaluar y del Caso Base}

La decisión estratégica de la empresa ficticia corresponde a la incorporación de la producción del diseño del Sr. Stuardo. Se debe evaluar si esto es recomendable o no. Esta decisión presenta un nivel de irreversibilidad que la convierte en una decisión estratégica con un impacto importante. La irreversibilidad se justifica en que es un producto de innovación de un segmento específico y acotado del mercado, y que no ha sido probado aún por los consumidores finales. Esto ocasiona que al comenzar a producir en volúmenes masivos no exista la opción de liquidar la producción completa a un competidor o empresa interesada, teniendo que hacerse cargo de todos los costos de mantención e inventario hasta su venta en el mercado o, alternativamente, de la destrucción de las unidades o su venta a un valor despreciable a empresas de reciclaje o similares. La importancia para la empresa del impacto que tiene la irreversibilidad de este negocio radica en el costo de oportunidad que se asocia a los gastos incurridos previo de tomar la decisión de retirarse de este negocio, es decir, a la posibilidad de haberse dedicado a un producto más estándar y con menor riesgo.

El caso base consiste en que la empresa ficticia no agrega el diseño del Sr. Stuardo a su línea de producción.
