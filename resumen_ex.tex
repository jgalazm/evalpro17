%!TEX root=main.tex
\section{Resumen ejecutivo}

El objetivo de este trabajo es valorizar la patente de invención de diseño de Sistema de Alcuza Integrada: INAPI registro N$^{o}$50.295 (22/10/2009). El dueño de esta es el Señor Rodrigo Stuardo. Para valorizar una patente se pueden aplicar distintas metodologías, en particular en el presente informe se profundiza en el método del VAN. Bajo ese contexto se evaluará el proyecto de manera tal como si se asumiera que ya existe una empresa con las capacidades para invertir y desarrollar el producto que se puede generar a partir de la patente, es decir, una alcuza integrada. El producto es entonces generado a partir de la patente de invención del cliente.

La empresa mencionada juega el rol de un posible comprador de la patente, el cual está interesado en utilizar el diseño para ampliar su negocio. Este negocio ha sido restringido de forma de operar  externalizando su producción para dedicarse principalmente a almacenar y producir el producto. La novedad y el acotado segmento del producto generado son los factores desafiantes que representan la irreversibilidad del proyecto. Sin embargo, el atractivo y la novedad funcional del dispositivo son ventajas competitivas cuyo impacto en el mercado se evalúan a través de la demanda.

Para estimar la demanda anual se utilizaron dos metodologías. En la primera, se obtuvo el porcentaje anual de familias que compran una alcuza, este valor fue de 28\% y se determinó que la participación de mercado de alcuza integrada es 0,7\%. Estos datos se determinaron a partir de información del Censo 2002, encuestas a público general, entrevistas a tiendas comerciales e información de catálogos de tiendas disponibles en internet. En la segunda metodología, se utilizó el modelo de Bass para estimar la demanda. En ambos métodos se consideró el ciclo de vida del producto, el cual tiene una duración de 3 años y se extendieron las metodologías a todo Chile.

En base a precios de alcuzas no tradicionales y al estado financiero de una tienda de retail se determinó que el precio de mercado de la alcuza integrada es \$9.356. La cantidad ofertada se determinó como una aproximación de la demanda total en los 3 años.

Los costos en los que incurre la empresa son en un comienzo desarrollar un producto a partir de un diseño. Posteriormente, se agregan los costos de producción, dado principalmente por la cristalería que fabrica la alcuza, costos de almacenaje, dado por la empresa de bodegaje, costos de despacho, empresa de transporte y despacho.

El VAN del proyecto da negativo con la primera metodología de estimación de la demanda y positivo para la segunda metodología. Para el caso base, el cual consiste en no realizar ninguna acción da igual a cero.

Se realizaron tres análisis de sensibilidad: a la tasa de descuento, a la participación de mercado del producto desarrollado a partir del Sistema de Alcuza Integrada para la primera metodología de estimación de la demanda, y al precio del producto para ambas metodologías.
