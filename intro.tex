%!TEX root=main.tex
Rodrigo Moisés Stuardo Carvajal, persona natural y arquitecto de la Universidad de Chile, "el cliente'', es dueño e inventor de la patente de invención con registro número 50.295 del Instituto Nacional de Propiedad Intelectual (INAPI), cuyo invento se denomina Sistema de Alcuza Integrada (SAI), que consiste en un nuevo sistema para organizar utensilios o envases dosificadores de especias líquidas y en grano que sirven para preparar, aliñar, condimentar y sazonar los alimentos. Se puede apreciar en la figura \ref{foto_alcuza} una imagen promocional de una alcuza fabricada a partir del SAI.

La génesis del SAI se encuentra en el año 2009 luego de identificar defectos en los sistemas tradicionales de alcuza y aceiteros, como por ejemplo el limitado control en las dosis servidas, goteo inevitable de su contenido, diseño sobrecargado, entre otros. Esto motiva la concepción del SAI y la consecuente construcción de prototipos para validar su recepción por parte del público, como los que se muestran en la figura \ref{prototipos} del anexo \ref{anexo:figuras} \footnote{También existe un video explicativo publicado en \url{https://www.youtube.com/watch?v=uLd1JcDwyxE}}

El objetivo del Sr. Stuardo es enriquecer su imagen y carrera profesional de arquitecto, diseñador e inventor por medio del reconocimiento de sus pares diseñadores y de autoridades relacionadas, como también logrando tener éxito en la evolución de su invento en un producto de mercado. Este objetivo motivó su participación en el  “1$^{o}$  Concurso EnVase a Chile” el año 2009, donde destacó como finalista, y en la “4$^{o}$ Feria Innova” de Rancagua, donde también fue seleccionado en el área de diseño. Es aquí donde el Sr. Rodrigo decide proteger su creación, presentando la solicitud para patentarla en el INAPI como patente de Diseño Industrial, donde se determina posteriormente el valor de originalidad tanto en el país como en el mundo, que permitió clasificarla como Patente de Invención con vigencia de 20 años desde el 2009.

Para lograr la transferencia de su diseño al mercado el Sr. Rodrigo desea vender los derechos de propiedad intelectual a una persona o empresa que posea la capacidad de producir y también distribuir modelos de alcuza basados en el SAI a los consumidores; él no posee interés en su fabricación ni distribución. Destaca aquí, en el año 2009, la ocasión más cercana a consolidar la venta de la patente, cuando se llegó a un acuerdo con la empresa Bogaris S.A.
\footnote{
Bogaris S.A. es una empresa trasnacional con presencia en España, Portugal, Rumania y Bulgaria que centra sus actividades en parques comerciales, instalaciones y aceites de oliva ultravirgen, ver:
 \url{http://wwww.bogaris.com/es/bogaris/nuestra_empresa/index.html}

 }
 en que le comprarían la patente a 7 o 14 millones de pesos dependiendo del resultado obtenido en el “1er Concurso EnVase a Chile”. Si bien el SAI logró ser finalista del concurso, la empresa Bogaris S.A no volvió a comunicarse con el Sr. Rodrigo y la venta no se concretó. Desde entonces, el Sr. Rodrigo mantiene la hipótesis de que es necesario generar una primera línea de producción de alcuzas basada en el SAI para demostrar el valor de su invención a los inversionistas.

Si bien el Sr. Stuardo ha intentado participar en otros concursos de diseño al pie cuáles y vender su patente a otras empresas, ha enfrentado otras barreras que se lo han impedido. Algunas de éstas son: no tener un producto terminado, no ser una empresa con personalidad jurídica, y lo más importante, no tener una noción más precisa del precio al cual debiera venderse la patente ni argumentos que lo respalden. El objetivo de este trabajo es apoyar al Sr. Stuardo en su proceso de transformación del (diseño del) SAI en un producto de mercado mediante la aplicación de técnicas de evaluación de proyectos que permitan valorizar su patente y así determinar un precio de mercado para la patente. Esta valorización consiste en una primera etapa para alcanzar el objetivo del Sr. Stuardo, con la cual se espera apoyar su decisión de venta a un cliente específico mediante conclusiones basadas en datos consistentes y metodologías validadas de economía e ingeniería industrial.

En la sección \ref{diagnostico} se describe más detalladamente la invención y el estado actual del proyecto en el proceso de venta, como también aspectos relevantes sobre la regulación de propiedad industrial vigente en Chile y el mundo. Esto permite caracterizar un diagnóstico del proyecto basado en los análisis de FODA y fuerzas de Porter, y definir objetivos específicos para este estudio. Posteriormente, en la sección \ref{metodologia}, se discute una metodología de evaluación basada en los posibles mercados donde podría ingresar el SAI y la información pública disponible. Finalmente, en la sección \ref{analisis}, se analizan los resultados obtenidos, lo que permite concluir con recomendaciones finales en la sección \ref{conclusiones}. Adicionalmente, se incorpora un glosario de términos relevantes en el anexo \ref{glosario}, para facilitar el entendimiento del lector.
