%!TEX root=main.tex
Se ha definido un modelo para la valorización de la patente de invención que indica los parámetros más relevantes a tener en consideración al momento de evaluar o ejecutar el proyecto.

La diferencia entre las metodologías en la estimación de los perfiles de demanda, permite identificar que el éxito del proyecto depende en gran medida del valor real y es en donde se debe hacer el mayor esfuerzo de estimación. Un análisis de mercado más detallado encargado a algún agente externo permitiría conocer con mayor precisión el valor esperado de la demanda, por lo que sería relevante conocer a priori el costo de realizar dicho estudio.

También es relevante en esta evaluación la cantidad inicial de unidades producidas ya que en ambos escenarios, la demanda total es menor a esta cantidad, actualmente se considera la restricción de pedido minimo del proveedor local, actualmente se espera la respuesta por parte de proveedores de China que tienen mayor holgura en dicha cantidad, pero con la consiguiente diferencia en la estructura de costos al considerarse por ejemplo costos de envío.


Por otro lado el presente trabajo se limita a la evaluación económica del proyecto considerando las variables más relevantes asociadas a este, sin embargo se deben considerar las distintas líneas de financiamiento para el cliente en particular ya que de conseguirse le permitirían completar el desarrollo del producto además de generar un MVP y con ello tener un parámetro más acotado para los costos y poder presentar el proyecto ante posibles compradores, inversionistas, patrocinadores o asociaciones con marcas.

Se recomienda la presentación de los resultados de esta evaluación a entidades que podrían estar interesadas en la compra de los de derechos de la patente. En conjunto con esto se recomienda al cliente finalizar el diseño de la tapa conectora para poder tener un diseño completo que presentar.
