\begin{itemize}
\item Alcuza:   (del árabe hispánico «alkúza», a su vez del árabe clásico «kūzah», y este del arameo «kūz[ā]», y este del persa «kuze») es una vasija para almacenar y administrar el aceite. El término alcuza se ha perdido en favor del más general aceitera, que puede denominar aceiteras, vinagreras y juegos de recipientes para aliñar las ensaladas.Pieza con dos frascos para aceite y vinagre).
\item Atomizar:         	Dividir algo en partes sumamente pequeñas, pulverizar.
\item Concepto:        	Idea que concibe o forma el entendimiento.
Conjunto contenedor	Un agregado de varias cosas que llevan o encierran dentro de sí a otras.
Diseño  Concepción original de un objeto u obra destinados a la producción en serie.
\item Diseño industrial:          	Proceso de diseño aplicado a los productos que se van a fabricar mediante técnicas de producción en masa.
Dispensador (distribuir) Dividir o repartir una cosa, señalando lo que corresponde a cada parte.
\item Dosificar:          	Dividir o graduar la cantidad o porción de algunas cosas.
\item Ensamble:        	Conjunto de piezas unidas, juntadas y/o ajustadas entre sí.
Envase. Recipiente o vaso en que se conservan y transportan ciertos géneros. Aquello que envuelve o contiene artículos de comercio u otros efectos para conservarlos o transportarlos.
\item Ergonomía:      	Estudio de la adaptación de las máquinas, muebles y utensilios a la persona que los emplea habitualmente, para lograr una mayor comodidad y eficacia.
\item Idea:   	Primero y más obvio de los actos del entendimiento, que se limita al simple conocimiento de algo. Imagen o representación que del objeto percibido queda en la mente.
\item Innovación:     	Es la creación o modificación de un producto o proceso y su introducción en un mercado que a su vez soluciona un problema de la técnica.
\item Integrado:       	Dicho de un conjunto de objetos: que constituyen un todo.
\item Invención:       	Es una solución nueva a un problema técnico, que genera actividad industrial, pudiendo dicha solución estar dada por un producto o un procedimiento. Para obtener una patente de invención, la invención debe ser nueva, inventiva y susceptible de aplicación industrial.
\item Maqueta:         	 Montaje funcional, a menor o mayor escala de un objeto, artefacto u edificio, realizada con materiales pensados para mostrar su funcionalidad, volumetría, mecanismos internos o externos o bien para destacar aquello que, en su escala real, una vez construido o fabricado, presentará como innovación o mejora.
Materialidad  	Superficie exterior o apariencia de las cosas. Se refiere al material básico de composición del objeto.
\item Packaging:       	Un recipiente o envoltura que contiene productos de manera temporal principalmente para agrupar unidades de un producto pensando en su manipulación, transporte y almacenaje.
\item Patente: Es un derecho de exclusividad, concedido por el Estado, para proteger y explotar una invención por el tiempo que determine la Ley.
\item PET:    	es un tipo de plástico muy usado en envases de bebidas y textiles.
\item Plástico: Dicho de ciertos materiales sintéticos: Que pueden moldearse fácilmente  y están compuestos principalmente por polímeros, como la celulosa.
\item Producción en Serie:    	Proceso en la producción industrial cuya base es la cadena de montaje o línea de ensamblado o línea de producción; una forma de organización de la producción que delega a cada trabajador una función específica y especializada en máquinas también más desarrolladas.
\item Producto de mercado:    Una opción elegible, viable y repetible que la oferta pone a disposición de la demanda, para satisfacer una necesidad o atender un deseo a través de su uso o consumo.
\item Prototipo:        	Un ejemplar o primer molde en que se fabrica una figura u otra cosa.
\item Pulverizar:       	Esparcir un líquido en partículas muy tenues, a manera de polvo
\item PVC:    	es el producto de la polimerización del monómero de cloruro de vinilo.
\item Recipiente.:    	Utensilio destinado a guardar o conservar algo.
\item Sistema industrial:        	Sistema que proporciona una estructura que agiliza la descripción, la ejecución y el planteamiento de un proceso industrial
Sistema Conjunto de cosas que relacionadas entre sí ordenadamente contribuyen a determinado objeto.
Tapa  	Pieza que cierra por la parte superior cajas o recipientes.
\item Utensilio de cocina:      	Es una herramienta que se utiliza en el ámbito culinario para la preparación de los platos.
\item Válvula: Mecanismo que regula el flujo de la comunicación entre dos partes de una máquina o sistema. Mecanismo que impide el retroceso de un fluido que circula por un conducto.
\item Vidrio:	Material duro, frágil y transparente o translúcido, sin estructura  cristalina, obtenido por la fusión de arena silícea con potasa y moldeable a altas temperaturas.
\end{itemize}
