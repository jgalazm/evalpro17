%!TEX root=main.tex
Según la metodología 1 el VAN del proyecto es $-\$16.580.118$, como el VAN es negativo podemos inferir que el proyecto no es rentable según la metodología 1, por otro lado el proyecto no tiene TIR ni período de payback.

Según la metodología 2 el VAN del proyecto es $\$33.833.974$, como el VAN es positivo podemos inferir que el proyecto es rentable según el modelo planteado. Su TIR y Payback son 20,27\% y 3 años respectivamente.

La diferencia entre las metodologías es la estimación de la demanda, siendo la segunda en la que se obtiene el mayor valor en ésta, lo que explica que se vendan más unidades y el VAN del proyecto sea positivo.

Esto revela la importancia que tiene lograr un buen posicionamiento del producto, debido a que hay una alta variación del valor del proyecto respecto a la demanda que se puede observar.

\subsection{Sensibilidad}

Se tomó la metodología 1 y se varió la cantidad de alcuzas no tradicionales con las que el producto derivado del SAI compartiría estante. Variar este parámetro se traduce en una variación de la participación de mercado, y es un indicador de qué tan demandado es el producto en comparación con otras alcuzas no tradicionales.

En la metodología 1, se estima que este parámetro es 17. El rango en que se varió en el análisis de sensibilidad fue entre 1 y 29, tomando solo valores discretos. Este rango se decició debido a que se desconoce qué tan demandado será el producto en comparación con su competencia más cercana de alcuzas no tradicionales. El detalle de los resultados se presenta en el Anexo \ref{AnSensibilidad1} y un gráfico resumen se presenta en la figura \ref{GraficoSensibilidad1}. Notemos que el VAN se hace 0 cuando este parámetro toma valores entre 13 y 14.

\subsection{Flexibilidad}

Un análisis de flexibilidad que se puede realizar es respecto a la empresa ficticia, dependiendo de si se ve obligada a comprar el lote completo de insumos para producir o si puede comprar la opción de comprar una fracción menor para así lanzar la primera producción, probar la recepción en el mercado, y luego decidir si pedir la otra fracción del lote. Llamamos a esta opción la de flexibilidad número 1. Esto se representa en la figura \ref{flexibilidad}, donde $q$ es la probabilidad de tener éxito con el primer volumen de producción y $r$ la probabilidad de tener éxito en el segundo, asumiendo que se decidió producir el primero.

\subsection{Sustentación}

La sustentación del proyecto está en que se trata de una invención, por lo que el producto final puede variar en esta idea y sus distintas versiones pueden perdurar en el mercado de forma que su ciclo de vida es mayor a la de un producto único. Por otro lado, como se consideró una empresa ficticia, ésta ya posee una estabilidad económica y una solidez, que permite aquellos periodos en que la demanda no se comportó como lo esperado y se pueda mantener el proyecto.
