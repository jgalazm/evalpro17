\subsection{Sobre la propiedad intelectual}

La propiedad intelectual es una rama del derecho que fomenta la innovación, la creación y la transferencia tecnológica mediante la definición de derechos exclusivos sobre las invenciones o creaciones a cambio de que estas sean dispuestas al público general y que pasen a ser parte del dominio público luego de un tiempo determinado.  Un tipo particular de propiedad intelectual corresponde a la propiedad industrial, que corresponde a los derechos que una persona física o jurídica puede tener sobre una invención, las cuales se clasifican en: patentes de invención, modelos de utilidad, marcas comerciales, colectivas, de certificación e indicaciones geográficas y denominaciones de origen. El organismo encargado de administrar y atender los servicios asociados a la propiedad industrial es el INAPI. Además, existe el Tribunal de Propiedad Industrial (TDPI) como órgano jurisdiccional e independiente, el cual atiende las apelaciones sobre las resoluciones dictadas por el Director del INAPI.

A diferencia de otras formas de propiedad industrial, las invenciones se describen como “toda solución a un problema de la técnica que origine un quehacer industrial” (Ley 19039, art. 31) y deben caracterizarse por su novedad, nivel inventivo y aplicación industrial. En este caso, una patente de invención otorga protección sobre los derechos del propietario por un total de 20 años en el territorio del país, acerca del objeto de invención definido en la patente. Adicionalmente, el dueño de la patente permite registrar la propiedad industrial en los más de 200 países adheridos a la Organización Mundial de Propiedad Intelectual (OMPI) de forma más expedita, gracias al Tratado de Cooperación en materia de Patentes (PCT).

Respecto a la amplitud de la protección que otorga una patente de invención, la Ley 19039 especifica en el artículo 49 que ésta se determina por el contenido de la sección “reivindicaciones” de la patente misma, dejando la interpretación de la misma dependiendo de lo que se especifique en la sección “memoria descriptiva”. En el caso del SAI,  esto viene determinado por dos partes que se pueden resumir como: (1) todo conjunto coaxial compuesto de dos frascos idénticos mediante una tapa doble interpuesta entre éstos y (2) los detalles geométricos  expresados en las figuras que acompañan a la patente. La descripción más precisa del documento original se encuentra en el anexo \ref{} \footnote{Para la interpretación de esta ley se está gestionando una reunión con expertos en Propiedad Industrial del Centro de Innovación Anacleto Angelini.}.


Para realizar la venta de una patente, es decir, de los derechos de propiedad asociados a una patente de invención, normalmente se debe contactar al dueño para negociar la transferencia de estos ante el organismo correspondiente. No existe un organismo oficial encargado de gestionar el mercado de patentes, y aunque existen iniciativas privadas para realizar subastas \textcolor{red}{anexo auctions}, no existe información histórica disponible sobre el precio acordado y la fecha en que se consolidó.


Finalmente, es interesante notar que si bien la alcuza es un producto específico dentro de un surtido de productos de utensilios de cocina, como se ve en las figuras \ref{alcuzas_tiempo}, según la OMPI se han solicitado en promedio 4 patentes por año, destacando el año
