\textbf{Paris.}

\begin{itemize}
\item Cantidad de diseños diferentes de alcuzas a la venta: 1.
\item El encargado de ventas indica que la venta de alcuzas es “casi nula” y recomendó buscar en tiendas especializadas o Falabella tal vez.
\end{itemize}

\textbf{Falabella.}

La vendedora indica distintas afirmaciones según su experiencia adquirida en el tipo que lleva trabajando en la tienda:
\begin{itemize}
\item Afirma que los clientes “compran harto para armar una casa o cuando están remodelando cocina por ejemplo”.
\item El perfil de clientes son “varones separados”, “parejas jóvenes argentinas”.
\item Para el caso de los hombres, estos buscan calidad más que precio. Al contrario, las mujeres priorizan el precio.
\item La vendedora lleva 4 meses trabajando y no han cambiado ninguna alcuza de las repisas.
\item En cuanto a las ventas afirma que “por mes venden aprox.15 (noviembre y diciembre), el resto del año a lo más 5 por mes”.
\item Afirma que “las alcuzas se venden más en época de navidad, como regalo. El resto del año es muy poco lo que se vende. Este año para navidad tuvieron un diseño de \$26.000 que se vendió harto,había una de \$15.000 también, el resto no se vendió”.
\item Principales marcas de alcuzas: Basement - Propaga.
\item La proporción de venta entre alcuzas tradicionales y no tradicionales es alrededor de 8:2.
\end{itemize}

\textbf{Ripley.}

\begin{itemize}
\item Cantidad de diseños diferentes de alcuzas a la venta: 1.
 \item La vendedora afirma que desde marzo que no ha vendido ninguna alcuza.
\end{itemize}

\textbf{Casa\&ideas.}

El vendedor responde según su experiencia en el tiempo que lleva trabajando en la tienda:

\begin{itemize}
\item Se vende más la alcuza compuesta por varios recipientes que la alcuza de una sola botella.
\item Aproximadamente de venden 20 unidades de alcuzas al mes.
\item ”El perfil de clientes que compran alcuzas es muy variado”.
\item Lleva dos meses trabajando en la tienda y ha visto el mismo modelo de alcuza durante todo ese tiempo.
\end{itemize}

\textbf{Kitchen republic.}

La vendedora responde según la experiencia adquirida durante el tiempo que lleva trabajando en la tienda.

\begin{itemize}
\item Venden aproximadamente 40 alcuzas al mes.
\item ”Lo compra todo tipo de gente”.
\end{itemize}

\textbf{Capdor.}

El vendedor responde según la experiencia adquirida durante el tiempo que lleva trabajando en la tienda.

\begin{itemize}
\item Cantidad de diseños diferentes de alcuzas a la venta: entre 6 y 8.
\item Se cambian los diseños de alcuzas a la venta 2 o 3 veces al año.
\item Al mes se venden entre 15 y 20 alcuzas.
\item El principal motivo de compra es por regalo o reposición.
\item La alcuza más vendida es un producto chileno, de Cristal Art.
\end{itemize}
